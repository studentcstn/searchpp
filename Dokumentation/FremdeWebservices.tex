\chapter{Fremde Webservices}
In diesem Kapitel wird die Verbindung zu fremden Webservices, Erfahrungen damit und eventuelle Probleme beschrieben.

\section{Amazon}
Von Amazon wurde die Amazon Product Advertising API verwendet. Diese API wurde ausgesucht um Produktpreise, Produktinformationen und Bewertungen abzurufen. Leider mussten wir feststellen, dass Amazon die Produktbewertungen �ber die API nicht weitergibt. Genaueres dazu bei den Erfahrungen.

\subsection{Verbindung}
Um die API zu verwenden wird ein Amazon Partnernet Account ben�tigt. Hat man dort ein Konto kann man sich einen Access Key generieren lassen. Dieser wird f�r Anfragen an die API ben�tigt. Um diese erfolgreich zu stellen, muss man folgende Schritte beachten um die Anfrage zu signieren:

\begin{itemize}
	\item Zeitstempel an die URL anh�ngen
	\item Kodierung von ',' und ':' in RFC 3986 Spezifikation
	\item Parameter/Wert Paare am '\&'-Zeichen Teilen
	\item Parameter/Wert Paare nach Byte-Wert sortieren
	\item Paare wieder mit '\&' zusammenf�gen
	\item 3 weitere Zeilen vor dem String anh�ngen
	\item Die Anfrage mit SHA256 �ber den AWS Secret Key verschl�sseln
	\item '+' und '=' Kodieren
	\item Berechneten Schl�ssel als Signatur an Anfrage anh�ngen
\end{itemize}
Genaueres ist \href{https://docs.aws.amazon.com/AWSECommerceService/latest/DG/rest-signature.html}{hier} beschrieben.

Um einzelne Produkte �ber die ASIN (eindeutige ID f�r jedes Produkt auf Amazon) zu suchen wurde in der Anfrage f�r den Parameter "'Operation"' der Wert "'ItemLookup"' und f�r "'IdType"' der Wert "'ASIN"' angegeben. F�r mehrere Produkte wurde f�r "'Operation"' der Wert "'ItemSearch"' verwendet. Der "'SearchIndex"' legt fest in welcher Kategorie die Produkte gesucht werden sollen. Mit der "'ResponseGroup"' kann die Antwort von Amazon eingeschr�nkt werden, um uninteressante Werte auszulassen.

F�r das Projekt waren folgende Werte von Amazon interessant:

\begin{itemize}
	\item ASIN - Die ID des Produkts auf Amazon
	\item Title - Der Produkttitel
	\item Condition - Artikelzustand
	\item LowestNewPrice - Der g�nstigste Preis f�r ein neues Produkt
	\item SalesRank - �hnlich wie Bestseller, Produkte mit geringen SalesRank wurden h�ufiger verkauft
	\item Manufacturer - Hersteller des Produkts
	\item Model - Modellbezeichnung des Produkts
	\item LargeImage - URL zu einem Produktfoto
	\item DetailPageURL - URL zum Produkt auf Amazon
	\item HasReviews - Gibt an ob das Produkt bewertet wurde
\end{itemize}

\subsection{Erfahrungen und Probleme}
Wie bereits erw�hnt gibt Amazon keine Bewertung �ber seine API weiter. Dieses war jedoch der Hauptgrund f�r uns Amazon zu verwenden. Da es auch keine andere API gibt, die kostenlos Bewertungen oder Testberichte von Produkten liefert, haben wir beschlossen, die Bewertungen von der Website direkt auszulesen. Da Amazon nach einigen Abfragen der Bewertungen mit dem gleichen User-Agent Captchas sendet, wurde eine Liste mit vielen User-Agents hinterlegt, die zuf�llig verwendet werden. Sollten dennoch Captchas zur�ckkommen wird das Produkt ignoriert, weshalb sich die Produktliste manchmal unterscheidet.
\\\\
Ein weiteres Problem ist, dass Amazon bei mehreren, aufeinanderfolgenden Anfragen manchmal den Fehler 503 mit der Information "'You are submitting requests too quickly. Please retry your requests at a slower rate."' zur�ckliefert. Diese tritt vor allem bei der Suche nach mehreren Produkten auf. Mit einer Anfrage bekommt man nur 10 Ergebnisse geliefert, man kann aber pro Anfrage eine Seitenzahl angeben. Diese kann bei der Suche in allen Kategorien maximal 5 sein. Um also 50 Ergebnisse zu erhalten m�ssen 5 Abfragen nacheinander erfolgen. Dabei bremst Amazon, durch den Fehler, die Anfragen teilweise stark aus.
\\\\
Die Dokumentation auf der Website k�nnte von der Navigation und Gliederung auch besser strukturiert sein. An manchen Stellen werden Beispiele beschrieben, die entweder nicht mehr funktionieren oder deren Funktion sonst nicht beschrieben wird.

\section{Ebay}
Von Ebay wurde die Finding API verwendet. Die API wurde ausgesucht, um weitere Preise und Angebote zu Produkten zu finden, die nach dem Algorithmus als "'empfohlen"' eingestuft wurden. Dazu z�hlen auch B-Ware und Gebrauchte Produkte (sofern der Benutzer das m�chte), die entweder als Auktion oder als Sofortkauf angeboten werden.

\subsection{Verbindung}
Zur Verwendung der API wird ebenfalls ein Key ben�tigt. Daf�r kann man sich einfach ein neues Ebay Konto anlegen, oder ein bestehendes verwenden. Mit einem Konto kann man sich dann Keys f�r den Sandbox Modus oder den Production Modus generieren lassen. Der Sandbox Modus erm�glicht es dem Benutzer mit einem virtuellen Kapital virtuelle Produkte zu kaufen. Dieser wurde von uns aber nicht benutzt.
F�r eine erfolgreiche Anfrage muss lediglich die App-ID, die mit dem Key generiert wurde angegeben werden.
\\\\
Leider konnten die Anfragen f�r eine Liste von Produkten und die Anfrage f�r ein einzelnes Produkt nicht �ber die selbe API erreicht werden. Beide API liefern jedoch �hnliche Antworten, die sich eigentlich nur durch Gro�-/Kleinschreibung der Attribute unterscheiden.

Folgende Informationen waren f�r das Projekt von Interesse:

\begin{itemize}
	\item ItemId - Die ID des Produkts auf Ebay
	\item Title - Der Titel des Angebots
	\item Condition - Der Artikelzustand, wird als ID geliefert, die viele Werte annehmen kann (siehe Enumeration Condition in searchpp.model.products)
	\item CurrentPrice - Der aktuelle Preis des Artikels
	\item ListingTyp - Art des Angbots (siehe Enumeration ListingType in searchpp.model.products)
	\item GalleryURL - URL zu einem Produktfoto
	\item ViewItemURL - URL zur Produktseite auf Ebay
\end{itemize}

Wichtig ist es bei den Anfragen noch die Global-ID bzw. Site-ID festzulegen. Diese gibt n�mlich an ob die Produkte z.B. auf ebay.de oder auf ebay.com gesucht werden.

\subsection{Erfahrungen und Probleme}
Die Verwendung von Ebay ist sehr einfach, vor allem im Vergleich mit Amazon. Man kann direkt ein Konto anlegen, ohne spezielle Anforderungen zu erf�llen oder sich zun�chst zu bewerben. Die Anfragen m�ssen ebenfalls nicht bearbeitet werden und k�nnen so direkt von Browser aus gestellt und getestet werden. Die Anzahl der Anfragen ist unbegrenzt.

Bei der Dokumentation sind keine Fehler aufgefallen und es wurden viele Beispiele eingearbeitet.
\section{Google}

