\chapter{Client}
\section{Verwendete Frameworks}
F�r das Design des Clients wurde Bootstrap verwendet. F�r die Webanwendung wurde
das Framework AngularJS benutzt. Der Client wurde mithilfe von AngularJS bzw.
ngRoute als Singlepage-Application aufgebaut. Dabei besitzt jede View einen
eigenen Controller, der bei der Navigation aktiviert wird. Im Controller werden
je nach Seite Daten geladen oder nach Benutzereingabe abgerufen.
Seiten�bergreifende Daten werden nur im \$rootScope gespeichert, weshalb ein
Neuladen der Seite alle Informationen l�scht. Wenn sich ein Benutzer �ber Google
anmeldet, wird er wieder auf die Startseite weitergeleitet und der Token in der
URL �bergeben. Dieser wird dann im \$rootScope gespeichert und an die Anfragen
angeh�ngt.

\chapter{Quellcode Funktion}

Im Verzeichnis `Client` ist der Quellcode des Clients.
\\
Im Verzeichnis `src/main/java/searchpp` befindet sich der Quellcode des Servers,
unten aufgef�hrt ist die Zuordnung der Vezeichnisse zu Funktionen:

\begin{lstlisting}
Main          - Start des Servers
/database     - Verbindung zur Datenbank, so wie abspeichern 
                und laden von Daten.
/localservice - Price history service f�r regelm��ige Preis 
                abfragen von beobachteten Produkten.
/model
  /config     - Zum Laden und verwenden von Zugangsdaten 
                zu den verschieden APIs
  /json       - Interfaces f�r Json Konvertierung
  /products   - Klassen f�r Produkte und Bewertungen
  /user       - Nutzer
/services     - Programmcode zum Download von Produkt- 
                information und Bewertung, so wie Zugriff 
                auf Google Kalender
/sites        - Restfull Schnittstellen
/utils        - Hilfsprogramme f�r Zugriff auf APIs und 
                Konfigurationen
\end{lstlisting}

Im root Verzeichnis der Software liegt noch die seachpp.conf. In dieser stehen
die Zugangsdaten zu den APIs.
