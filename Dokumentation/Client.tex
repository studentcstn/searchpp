\chapter{Client}
\section{Verwendete Frameworks}
F�r das Design des Clients wurde Bootstrap verwendet. F�r die Webanwendung wurde das Framework AngularJS benutzt. Der Client wurde mithilfe von AngularJS bzw. ngRoute als Singlepage-Application aufgebaut. Dabei besitzt jede View einen eigenen Controller, der bei der Navigation aktiviert wird. Im Controller werden je nach Seite Daten geladen oder nach Benutzereingabe abgerufen. Seiten�bergreifende Daten werden nur im \$rootScope gespeichert, weshalb ein Neuladen der Seite alle Informationen l�scht. Wenn sich ein Benutzer �ber Google anmeldet, wird er wieder auf die Startseite weitergeleitet und der Token in der URL �bergeben. Dieser wird dann im \$rootScope gespeichert und an die Anfragen angeh�ngt.

\chapter{Projektverzeichniss}

\section{Server}
Beim Abruf von Amazon Produkt Bewertungen kann es passieren, dass Amazon das System als Roboter erkennt und zeigt ein Captcha an.
Mit vielen verschieden User-Agends wird versucht, diese zu umgehen und trotzdem an Daten zu kommen.


Der Aufbau von Search++ ist wie folgt:

Main          - Start des Servers
/database     - Verbindung zur Datenbank, so wie abspeichern und laden von Daten.
/localservice - Price history service für regelmäßige preis abfragen von beobachteten Produkten.
/model
  /config     - Zum laden und verwenden von Zugangs Daten zu den verschieden APIs
  /json       - interfaces für Json Konvertierung
  /products   - Klassen für Produkte und Bewertungen
  /user       - Nutzer
/services     - Programmcode zum Download von Produktinformation und Bewertung, so wie Zugriff auf Google Kalender
/sites        - Restfull Schnittstellen
/utils        - Hilfsprogramme für Zugriff auf APIs und Konfigurationen

Im root Verzeichnis der Software liegt noch die seachpp.conf. In dieser stehen die Zugangs Daten zu den APIs.