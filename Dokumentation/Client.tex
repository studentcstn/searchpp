\chapter{Client}
F�r das Design des Clients wurde Bootstrap, f�r die Webanwendung
das Framework AngularJS benutzt. Der Client wurde mithilfe von AngularJS bzw.
ngRoute als Singlepage-Application aufgebaut. Dabei besitzt jede View einen
eigenen Controller, der bei der Navigation aktiviert wird. Im Controller werden
je nach Seite Daten geladen oder nach Benutzereingabe abgerufen.
Seiten�bergreifende Daten werden nur im \$rootScope gespeichert, weshalb ein
neu Laden der Seite alle Informationen l�scht. Wenn sich ein Benutzer �ber Google
anmeldet, wird er wieder auf die Startseite weitergeleitet und der Token in der
URL �bergeben. Dieser wird dann im \$rootScope gespeichert und an die Anfragen
angeh�ngt.


